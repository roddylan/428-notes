\documentclass{article}
\usepackage{textcomp, gensymb}
\usepackage{utf8add}
\usepackage[most]{tcolorbox}
\usepackage{hyperref}
\usepackage{cleveref}
\usepackage{amsmath}
\usepackage{amssymb}
\usepackage{tcolorbox}
\usepackage{outlines}
\usepackage{enumitem}
\usepackage{graphics}
\usepackage{bbm}
% \usepackage{hhline}

\hypersetup{
    colorlinks,
    citecolor=black,
    filecolor=black,
    linkcolor=black,
    urlcolor=black
}
\DeclareMathOperator*{\argmax}{arg\,max}
\DeclareMathOperator*{\argmin}{arg\,min}

% \setenumerate[1]{label=\Roman*.}
% \setenumerate[2]{label=\Alph*.}
% \setenumerate[3]{label=\roman*.}
% \setenumerate[4]{label=\alph*.}

\title{CMPUT 428: 3D Modeling}
\author{Roderick Lan}
% \date{January 11, 2024}
\date{}

\usepackage{natbib}
\makeatletter
% \crefformat{tcb@cnt@Example}{example~#2#1#3}
% \Crefformat{tcb@cnt@Example}{Example~#2#1#3}
\makeatother
\newtcbtheorem[auto counter, number within = subsection]
{definition}{Definition}{%                                                        
  breakable,
  fonttitle = \bfseries,
  colframe = blue!75!black,
  colback = blue!10
}{def}

\makeatother
\newtcbtheorem[auto counter, number within = subsection]
{example}{Example}{%                                                        
  breakable,
  fonttitle = \bfseries,
  colframe = orange!75!black,
  colback = orange!10
}{ex}

\makeatother
\newtcbtheorem[auto counter, number within = subsection]
{expln}{Expln}{%                                                        
  breakable,
  fonttitle = \bfseries,
  colframe = red!75!black,
  colback = red!10
}{exp}

\makeatother
\newtcbtheorem[auto counter, number within = subsection]
{ovr}{Overview}{%                                                        
  breakable,
  fonttitle = \bfseries,
  colframe = blue!75!black,
  colback = blue!10!white!50
}{ovr}


\makeatother
\newtcbtheorem[auto counter, number within = subsection]
{refer}{Reference}{%                                                        
  breakable,
  fonttitle = \bfseries,
  colframe = red!75!black,
  colback = red!10
}{refer}
\begin{document}

\makeatother
\newtcbtheorem[auto counter, number within = subsection]
{thm}{Thm}{%                                                        
  breakable,
  fonttitle = \bfseries,
  colframe = orange!75!black,
  colback = orange!10
}{thm}

% \setcounter{section}{1}
\maketitle
\tableofcontents
\break
% \section*{Lecture 1}

\section{Lecture - Mar 12}
Structure from Silhouette
\begin{list}{}{}
  \item Get cone ray from silhouette
\end{list}

\subsection{Incremental Free Space Carving}
Triangulate sparse point cloud: remove tetrahedrons/triangles + remake w/ points

\subsection{3D modeling system}
online, incremental handling of new info events
\\
works with sparse point clouds (good for vision/feature based methods)
\\
models coarse


\subsection{3 Tier Model}
Macro, Meso, Micro model
\\
refine geometry w/ coarse model as prior
\noindent
Multi Tiered Models:
\begin{outline}
  \1 Commonly:
    \2 2 Tiers: 3D geom and appearance (texture mapping)
    \2 Used in graphics applications, recovered from vision applications
  \1 3 Tier:
    \2 Macro - scene geometry (triangulation map)
    \2 Meso - fine scale geometric detail (displacement map)
    \3 Micro - fine scale geometry/reflectance (texture map)
  \1 Captured via sequential refinement
\end{outline}

\subsection{Multiscale Model}
Geometry alone doesnt solve modeling, need multiscale model
\\
Need
\begin{enumerate}
  \item Geometry
  \item Depth
  \item Dynamic Texture
\end{enumerate}
$\to$ Rendering
\\
Use image derivatives (know lighting changes, position of view, etc.) in forward way to render a diff. img (helps get photorealism)


\subsection{Capgui}
\subsubsection*{Step 1 - Calibration}
\subsubsection*{Step 2 - Segmentation}
Get rid of background
\subsubsection*{Step 3 - Shape From Silhouette}
8-60 imgs
\\
multiple views of same object $\to$ intersect \textbf{gneeralized cones} generated by each img
to build a volume (guaranteed to contain object)
\\
limiting smallest vol. obtainable in this way is known as the \textbf{visual hull} of the object


\subsubsection{SFS methods}
Voxel based (use voxel grid rep.)
\begin{list}{}{}
  \item inaccurate
  \item triangulate w/ marching cubes algo
\end{list}
Image ray based (use image rays)
\begin{list}{}{}
  \item accurate
\end{list}
Axis aligned (use rectlinear rays (instead of camera rays), mark 'cut' points of 
image rays)
\begin{list}{}{}
  \item moderately accurate
  \item fast
  \item marching intersections algo
  \item (mix of img ray and voxel based)
\end{list}

\subsubsection*{Step 4 - Phototextures + Texture Mapping}
For each triangle in model, establish corresponding region in the phototextures
\\
\textbf{Difficulties:}
\begin{outline}
  \1 Tedious to sepcify texture coords. for every triangle
  % \1 Acqui
\end{outline}

\subsubsection{Common Text. Coord. Mappings}
Orthogonal
\\
Cylindrical
\\
Spherical
\\
Perspective Projection
\\
Texture Chart (ie. text. split + flatten; cut object into pieces and map textures to each piece (piecwise planner))

\subsubsection{Advanced Texture Splitting and Mapping}
\textbf{Floating Planes} Method
\begin{outline}
  \1 split into dozen - several dozen perspective mappings
  \1 union of persp. planes accurately represent obj
\end{outline}
\noindent
\textbf{LCSM (Least Squares Conformal Mapping)}
\begin{outline}
  \1 least square (locally) preserve orthogonality
\end{outline}




\subsubsection*{Step 6 - Texture Basis Computation}


\noindent\rule{\textwidth}{0.1pt}

\subsection{Dyntex Theory}





\end{document}

