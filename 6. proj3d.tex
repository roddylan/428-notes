\documentclass{article}
\usepackage{textcomp, gensymb}
\usepackage{utf8add}
\usepackage[most]{tcolorbox}
\usepackage{hyperref}
\usepackage{cleveref}
\usepackage{amsmath}
\usepackage{amssymb}
\usepackage{tcolorbox}
\usepackage{outlines}
\usepackage{enumitem}
\usepackage{graphics}
\usepackage{bbm}
\hypersetup{
    colorlinks,
    citecolor=black,
    filecolor=black,
    linkcolor=black,
    urlcolor=black
}
\DeclareMathOperator*{\argmax}{arg\,max}
\DeclareMathOperator*{\argmin}{arg\,min}

% \setenumerate[1]{label=\Roman*.}
% \setenumerate[2]{label=\Alph*.}
% \setenumerate[3]{label=\roman*.}
% \setenumerate[4]{label=\alph*.}

\title{CMPUT 428: 3D Projective Geometry}
\author{Roderick Lan}
% \date{January 11, 2024}
\date{}

\usepackage{natbib}
\makeatletter
% \crefformat{tcb@cnt@Example}{example~#2#1#3}
% \Crefformat{tcb@cnt@Example}{Example~#2#1#3}
\makeatother
\newtcbtheorem[auto counter, number within = subsection]
{definition}{Definition}{%                                                        
  breakable,
  fonttitle = \bfseries,
  colframe = blue!75!black,
  colback = blue!10
}{def}

\makeatother
\newtcbtheorem[auto counter, number within = subsection]
{example}{Example}{%                                                        
  breakable,
  fonttitle = \bfseries,
  colframe = orange!75!black,
  colback = orange!10
}{ex}

\makeatother
\newtcbtheorem[auto counter, number within = subsection]
{expln}{Expln}{%                                                        
  breakable,
  fonttitle = \bfseries,
  colframe = red!75!black,
  colback = red!10
}{exp}

\makeatother
\newtcbtheorem[auto counter, number within = subsection]
{ovr}{Overview}{%                                                        
  breakable,
  fonttitle = \bfseries,
  colframe = blue!75!black,
  colback = blue!10!white!50
}{ovr}


\makeatother
\newtcbtheorem[auto counter, number within = subsection]
{refer}{Reference}{%                                                        
  breakable,
  fonttitle = \bfseries,
  colframe = red!75!black,
  colback = red!10
}{refer}
\begin{document}

\makeatother
\newtcbtheorem[auto counter, number within = subsection]
{thm}{Thm}{%                                                        
  breakable,
  fonttitle = \bfseries,
  colframe = orange!75!black,
  colback = orange!10
}{thm}

% \setcounter{section}{1}
\maketitle
\tableofcontents
\break
% \section*{Lecture 1}

\section{Lecture - Feb 13}
lec06Proj3D.pdf\\[5pt]
Points:
\[
  \mathbf X = (x_1, x_2, x_3, x_4); x_4 \ne 0
\]
Point at inf:
\[
  (x_1, x_2, x_3, 0)
\]
Planes:
\[
  \mathbf \pi = (\pi_1, \pi_2, \pi_3, \pi_4)
\]
Plane at inf:
\[
  \mathbf \pi^\top \mathbf X = 0
\]
proj transform $\to$ $4\times 4$ nonsingular matrix $H$\\
Point Transformation: $X' = HX$
\\
Plane Transformation: $\pi' =H^{-\top }\pi$
\\[5pt]
3D point:\\
$(X,Y,Z)^\top$ in $\mathbb R^3$
\\
$X = (X_1, X_2, X_3, X_4)$ in $P^3$
\\[5pt]
Euclidean Rep:
% \\
\[
  n^\top \tilde X + d = 0
\]
$
n = (\pi_1, \pi_2, \pi_3)^\top
$, $\tilde X = (X,Y,Z)^\top $
\\
$\pi_4 = d$, $X_4 = 1$
\\
$d$ is distance
\subsection{Planes from Points}
Solve $\pi$ for:
\[
  \mathbf X_1^\top \pi = 0, \mathbf X_2^\top \pi = 0, \text{ and }, \mathbf X_3^\top \pi = 0
\]
(use SVD)\\
or implicitly \textbf{Coplanarity Condition} (check slides)
\\[5pt]
Canonical Form:
\\
If we have a plane, we can get nullspace span $\mathbf M= \begin{bmatrix}
  p \\ I
\end{bmatrix}$ 
\[
  p = \begin{bmatrix}
    -\frac{b}{a}, -\frac{c}{a}, -\frac{d}{a}
  \end{bmatrix}^\top
\]

\subsection{Points from Planes}
Solve $\mathbf X$ from
\[
  \pi_1 ^\top \mathbf X=0, \pi_2 ^\top \mathbf X=0, \text{ and } \pi_3 ^\top \mathbf X=0
\]

\subsection{Lines}
Join of two points ($A,B$)
\[
  W =\begin{bmatrix}
    A^\top \\
    B^\top
  \end{bmatrix}, \lambda A + \mu B
\]
\\
Intersection of two planes ($P,Q$)
\[
  W^* =\begin{bmatrix}
    P^\top \\
    Q^\top
  \end{bmatrix}, \lambda P + \mu Q \text{ "pencil of planes"}
\]
pencil of planes - all the planes that could generate an intersection of this line
\[
  W^* W^\top = WW^{*\top} = 0_{2\times 2}
\]
\\[10pt]
Define basis/direction vectors as pure direction ($x_4=0$) rather than a finite point
\\
$W$ is $2\times 4$

\subsection{Points, lines, planes}
Join of point $X$ and line $W$ is plane $\pi$
\[
  \mathbf M = \begin{bmatrix}
    W \\ X^\top
  \end{bmatrix} \ \ \ \mathbf M \pi = 0
\]
Intersection of line $W$ with plane $\pi$ is point $X$
\[
  \mathbf M = \begin{bmatrix}
    W^* \\ \pi ^\top
  \end{bmatrix} \ \ \ \mathbf MX = 0
\]

\subsection{Affine space}
Difficulties with a projective space:
\begin{outline}
  \1 Nonintuitive notion of direction
  \2 Parallelism not represented
  \2 Restore parallelism $\to$ need to know/fix some affine property
  \1 Infinity not distinguished
  \1 No notion of "inbetweenness"
  \2 Projective lines are topologically circular
  \1 Only cross-ratios are available
  \2 Ratios are required for many practical tasks
\end{outline}
% \\[5pt]
Solution:\\
Find the point at infinity
\[
  \pi_\infty = (0,0,0,1)
\]
Transform model to give $\pi_\infty$ canonical coords
\\
2D analogy: fix horizontal line $\mathbf l_\infty = (0,0,1)$
\\[5pt]
\noindent\rule{\textwidth}{0.5pt}
\subsubsection{Proj to Affine}
Get "parallel" lines in img, find projective transformation that would make 
lines parallel\\[5pt]
\noindent\rule{\textwidth}{0.5pt}
\\[15pt]
Affine transformation:
\[
  H_A = \begin{bmatrix}
    a_{11} & a_{12} & a_{13} & a_{14} \\
    a_{21} & a_{22} & a_{23} & a_{24}\\
    a_{31} & a_{32} & a_{33} & a_{34} \\
    0 & 0 & 0 & 1
  \end{bmatrix}
\]
12 DOF, $\pi_\infty$ unchanged
\\
Invariants:
\begin{list}{}{}
  \item Ratio of lengths on a line
  \item Ratios of angles
  \item Parallelism
\end{list}

\subsection{Metric Space}
Metric transformation (similarity)
\[
  H_S = \begin{bmatrix}
    sR & \mathbf t\\
    \mathbf O^t & 1
  \end{bmatrix}
\]
7DOF, maps absolute conic to itself
\\
Invariants:
\begin{list}{}{}
  \item Length ratios
  \item Angles
  \item Absolute conic
\end{list}
Highest level of geometric structure that can be retrived from images 
(without yard stick/ruler)

\begin{ovr}
  {Main Takeaway of 428}{}
  "Stratified Geometry" vs Euclidean Geometry
\end{ovr}
\noindent
Euclidean geometry is not the only repr for:
\begin{enumerate}
  \item building models from imgs
  \item building imgs from models
\end{enumerate}
When 3D realism is the goal; how can we effectively build and use projective,
affine, metric representations.











\section{Lecture - Feb 15}
lec08Proj, 
lec06Proj3D;
\textbf{READ SLIDES!!!!!!!!!}\\[5pt]
Projections - use similar triangles
\\
Camera matrix\[
  \begin{bmatrix}
    U \\ V \\ W
  \end{bmatrix}
  = \begin{bmatrix}
    1 & 0 & 0 & 0\\
    0 & 1 & 0 & 0\\
    0 & 0 & 1/f & 0
  \end{bmatrix}
  \begin{bmatrix}
    X \\ Y \\ Z \\ W
  \end{bmatrix}
\]
\[
  \begin{bmatrix}
    U \\ V\\ W
  \end{bmatrix}\approx
  \begin{bmatrix}
    f & 0 & 0 & 0\\
    0 & f & 0 & 0\\
    0 & 0 & 1 & 0
  \end{bmatrix}
  \begin{bmatrix}
    X \\ Y \\ Z \\ W
  \end{bmatrix}
\]
\[
  (U,V,W) \to \left ( \frac{U}{W}, \frac{V}{W} \right ) = (u,v)
\]
\\[10pt]
when coord sys not aligned, $P$ becomes arbitrary singular or nonsingular matrix
\[
  x = PX
\]


\subsection{Intrinsic Parameters}
Conversion from metric to pixel coords (and reverse)
\[
  x_{\mathrm{mm}} = -(x_{\mathrm{pix}}- o_x )s_x 
\]
\[
  y_{\mathrm{mm}} = -(y_{\mathrm{pix}}- o_y )s_x 
\]

\[
  \begin{bmatrix}
    x \\ y \\ w
  \end{bmatrix}_{\mathrm{pix}}
  =
  \begin{bmatrix}
    -f/s_x & 0 & o_x \\
    0 & -f/s_y & o_y \\
    0 & 0 & 1
  \end{bmatrix}
  \begin{bmatrix}
    x \\ y \\ w
  \end{bmatrix}_{\mathrm{mm}} = \mathbf M_{\mathrm{int}}\mathbf p
\]
\begin{expln}
  {}{}
  $f$ is focal length (mm); $s$ is scale factor (aspect ratios), $o$ center coords
\end{expln}


\subsection{Relative Location (cam/laser)}
create homog transforms that represent the alignments;\\
camera interal params from manufacturers\\
first matrix via calibration eq.\\
Internal params = known information; need to find ext. params (unknown)


\subsection{Diff Cam Models}
Orthographic $\to$ pure parallel projection\\
Weak Perspective $\to$ parallel projection onto imaginary plane (obj plane)
and then perspective transformation. (single scaling factor)
\\
Paraperspective $\to$ 
\\
Perspective $\to$

\subsection{Weak Perspective}
Uses a scaling factor $T$
\[
  u = Tx, v = Ty, T=f/Z
\]
$Z$ is the \textbf{average} depth

\[
  \begin{bmatrix}
    U \\ V\\ W
  \end{bmatrix} =
  \begin{bmatrix}
    1 & 0 & 0 & 0
    \\
    0 & 1 & 0 & 0\\
    0 & 0 & 0 & f/Z^*
  \end{bmatrix}
  \begin{bmatrix}
    X \\ Y \\ Z \\ T
  \end{bmatrix}
\]
$Z^*$ fixed val, usually mean/centroid distance


\subsection{Full Affine Linear Camera}
factor into rotation part and scale part (?)


\subsection{Camera Models}




\end{document}

