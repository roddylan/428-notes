\documentclass{article}
\usepackage{textcomp, gensymb}
\usepackage{utf8add}
\usepackage[most]{tcolorbox}
\usepackage{hyperref}
\usepackage{cleveref}
\usepackage{amsmath}
\usepackage{amssymb}
\usepackage{tcolorbox}
\usepackage{outlines}
\usepackage{enumitem}
\usepackage{graphics}
\usepackage{bbm}
\hypersetup{
    colorlinks,
    citecolor=black,
    filecolor=black,
    linkcolor=black,
    urlcolor=black
}
\DeclareMathOperator*{\argmax}{arg\,max}
\DeclareMathOperator*{\argmin}{arg\,min}

\setenumerate[1]{label=\Roman*.}
\setenumerate[2]{label=\Alph*.}
\setenumerate[3]{label=\roman*.}
\setenumerate[4]{label=\alph*.}

\title{CMPUT 428: Tracking}
\author{Roderick Lan}
% \date{January 11, 2024}
\date{}

\usepackage{natbib}
\makeatletter
% \crefformat{tcb@cnt@Example}{example~#2#1#3}
% \Crefformat{tcb@cnt@Example}{Example~#2#1#3}
\makeatother
\newtcbtheorem[auto counter, number within = subsection]
{definition}{Definition}{%                                                        
  breakable,
  fonttitle = \bfseries,
  colframe = blue!75!black,
  colback = blue!10
}{def}

\makeatother
\newtcbtheorem[auto counter, number within = subsection]
{example}{Example}{%                                                        
  breakable,
  fonttitle = \bfseries,
  colframe = orange!75!black,
  colback = orange!10
}{ex}

\makeatother
\newtcbtheorem[auto counter, number within = subsection]
{expln}{Expln}{%                                                        
  breakable,
  fonttitle = \bfseries,
  colframe = red!75!black,
  colback = red!10
}{exp}

\makeatother
\newtcbtheorem[auto counter, number within = subsection]
{ovr}{Overview}{%                                                        
  breakable,
  fonttitle = \bfseries,
  colframe = blue!75!black,
  colback = blue!10!white!50
}{ovr}


\makeatother
\newtcbtheorem[auto counter, number within = subsection]
{refer}{Reference}{%                                                        
  breakable,
  fonttitle = \bfseries,
  colframe = red!75!black,
  colback = red!10
}{refer}
\begin{document}

% \setcounter{section}{1}
\maketitle
\tableofcontents
\break
% \section*{Lecture 1}

\section{Lecture - Jan 30}
lec06GeomIntro.pdf\\
Image prediction is nonlinear.
(Pinhole) Camera collects light rays, projected onto plane (upside down).
\[
    (x,y,z ) \to \left ( f\frac{x}{z}, f \frac{y}{z}\right )
\]
or Projectively: $x = PX$\\
3D $\to$ 2D plane.\\
Choice (and positioning) of coordinate system very important, 
can simplify problem a lot. (we project img onto new coordinate system?)

\subsection{Matrix Rep. + Homogenous Coords}
Chain transformations together to build total transformation.
Good if all transformations can be represented as matrix mult. 
(comb. of transformation only involves mult. of matrices); Affine 
involves addition and multiplication. Translations don't have $2\times 2$, 
use Homogenous Coords to get $3\times 3$ matrix rep for transformation. 
\\
Append extra coords to 2D:
\[
    \begin{bmatrix}
        x' \\ y'
    \end{bmatrix} = \begin{bmatrix}
        x \\ y
    \end{bmatrix} + \begin{bmatrix}
        \Delta x \\ \Delta y
    \end{bmatrix}
    \Rightarrow
    \begin{bmatrix}
        x'\\
        y'\\
        1
    \end{bmatrix} = 
    \begin{bmatrix}
        1 & 0 & \Delta x\\
        0 & 1 & \Delta y\\
        0 & 0 & 1
    \end{bmatrix} \begin{bmatrix}
        x\\y\\1
    \end{bmatrix}
\]
$\mathbb R^4$ space sufficient for all Euclidean operaitons. 

\subsubsection{General Case}
Homog coord corresponding to $(x,y,z)$ is the triple $(x_h, y_h, z_h, w)$ s.t.
\begin{flalign*}
    x_h &= wx\\
    y_h &= wy\\
    z_h &= wz
\end{flalign*}
Initially set $w=1$. May perform differently w/ transformations (?)
\subsubsection{Transformations}
Translation:
\[
    P' = \begin{bmatrix}
        x' \\ y' \\ z' \\ w 
    \end{bmatrix} = \begin{bmatrix}
        1 & 0 & 0 & T_x\\
        0 & 1 & 0 & T_y\\
        0 & 0 & 1 & T_z\\
        0 & 0 & 0 & 1\\
    \end{bmatrix} \begin{bmatrix}
        x \\ y\\ z \\ w
    \end{bmatrix}
\]
Scaling:
Translation:
\[
    P' = \begin{bmatrix}
        x' \\ y' \\ z' \\ w 
    \end{bmatrix} = \begin{bmatrix}
        s_x & 0 & 0 & 0\\
        0 & s_y & 0 & 0\\
        0 & 0 & s_z & 0\\
        0 & 0 & 0 & 1\\
    \end{bmatrix} \begin{bmatrix}
        x \\ y\\ z \\ w
    \end{bmatrix}
\]
Using general idea of Homog. Transformation:
\[
    p' = \begin{bmatrix}
         &R & &T\\
        0 &0 & 0 & 1
    \end{bmatrix}
\]
Inverse:
\[
    p = \begin{bmatrix}
        &R' & &-R'T\\
        0 &0 & 0 & 1
    \end{bmatrix}
\]
(gen. rotation in slides (42))

\subsection{Rotation abt Specified Axis}
Useful to rotate about any axis in 3D space $\to$ compose 7 elementary transformations.
\\
Issue w/ just rotating: rotates about origin (object 'translates' and rotates).
Need to center object and align w/ coordinate system. (often times rotate on z axis).
\begin{center}
    \includegraphics*[width=.8\textwidth]{"imgs/rotation spec axis.png"}
\end{center}

\subsection{Basic Transformations}
Translation, scaling, etc.\\
Scaling be rewritten for fixed point.

\subsection{Camera Matrix}
\[
    (U,V,W) \to \left ( \frac{U}{W}, \frac{V}{W}\right ) = (u,v)
\]
\begin{expln}
    {}{}
    $W$ is often called the \textbf{scale factor}
\end{expln}\noindent
Homog Coords for 3D:
\[
    \begin{bmatrix}
        U \\ V \\ W
    \end{bmatrix} = \begin{bmatrix}
        1 & 0 & 0 & 0\\
        0 & 1 & 0 & 0\\
        0 & 0 & \frac{1}{f} & 0
    \end{bmatrix}\begin{bmatrix}
        X\\Y\\Z\\T
    \end{bmatrix}
\]
coord sys not aligned:
\begin{list}{-}{}
    \item projective (proj coords $\times$ arbitrary ($3\times4$) matrix $P$)
        \\$x=PX$
    \item euclidean\\$x=[R|T]X$
\end{list}
\begin{expln}
    {}{}
    Projective geometry preserves collinearity
\end{expln}\noindent

\subsection{Intersection (preview)}
can extend when we know img in plane and where camera is geometrically.
\\
need tracking for complex imgs (very important for correspondence)

\subsection{SFM (preview)}
Only have (unorganized) imgs, dont know where they were taken from. Multiple
images of same 3d object is redundant (?)   

\subsection{N-view geometry; Affine Factorization}
stack points in measurement matrix $W$, perform SVD (heath ch3), get
mathematically equivalent camera position ($\hat M$) and points ($\hat X$)
\\
$W = UDV^\top$ \\
$\hat W = U_{2m\times 3}D_{3\times 3}V^\top_{n\times 3} = \hat M \hat X$
\[
    W = \begin{bmatrix}
        \tilde x_1^1 & \cdots & \tilde x_n^1 \\
        \vdots & \ddots & \vdots\\
        \tilde x_1^m & \cdots & \tilde x_n^m
    \end{bmatrix} = \begin{bmatrix}
        M^1 \\ \vdots \\ M^m
    \end{bmatrix}
    \begin{bmatrix}
        x_1 & \cdots & x_n
    \end{bmatrix}
\]

\subsection{Limitations in images}
distant objects smaller; img prediction can rearrange position and scale arbitrarily
(can't trust size relationships)\\
Visual ambiguity.
\\[10pt]
Parallel lines meet at a point in projective geometry (vanishing point)
\subsection{Invariants}
Things that stay the same after every projective transform
\begin{list}{-}{}
    \item points map to points
    \item intersections are preserved (not necessarily at the same real/metric distances)
    \item lines map to lines
    \item collinearity preserved
    \item ratio of ratios (cross ratio); relative distances (no abs. distance)
    \item horizon
\end{list}


\subsection{Vanishing points}
Each set of parallel ines (direction) meets at a diff. point - \textbf{vanishing point}
for this direction.
\\
Sets of parallel lines on same plane lead to \textbf{collinear} vanishing point (\textbf{horizon} for that plane)

\subsection{Geometric Properties of Projection}
\begin{list}{-}{}
    \item points go to points
    \item lines go to lines
    \item (mathematical) planes go to whole img
    \item polygons go to polygons
\end{list}
Degenerate cases:
\begin{list}{-}{}
    \item line through focal point to point
    \item plane through focal point to line
\end{list}



\end{document}