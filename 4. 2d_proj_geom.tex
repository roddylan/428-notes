\documentclass{article}
\usepackage{textcomp, gensymb}
\usepackage{utf8add}
\usepackage[most]{tcolorbox}
\usepackage{hyperref}
\usepackage{cleveref}
\usepackage{amsmath}
\usepackage{amssymb}
\usepackage{tcolorbox}
\usepackage{outlines}
\usepackage{enumitem}
\usepackage{graphics}
\usepackage{bbm}
\hypersetup{
    colorlinks,
    citecolor=black,
    filecolor=black,
    linkcolor=black,
    urlcolor=black
}
\DeclareMathOperator*{\argmax}{arg\,max}
\DeclareMathOperator*{\argmin}{arg\,min}

\setenumerate[1]{label=\Roman*.}
\setenumerate[2]{label=\Alph*.}
\setenumerate[3]{label=\roman*.}
\setenumerate[4]{label=\alph*.}

\title{CMPUT 428: Tracking}
\author{Roderick Lan}
% \date{January 11, 2024}
\date{}

\usepackage{natbib}
\makeatletter
% \crefformat{tcb@cnt@Example}{example~#2#1#3}
% \Crefformat{tcb@cnt@Example}{Example~#2#1#3}
\makeatother
\newtcbtheorem[auto counter, number within = subsection]
{definition}{Definition}{%                                                        
  breakable,
  fonttitle = \bfseries,
  colframe = blue!75!black,
  colback = blue!10
}{def}

\makeatother
\newtcbtheorem[auto counter, number within = subsection]
{example}{Example}{%                                                        
  breakable,
  fonttitle = \bfseries,
  colframe = orange!75!black,
  colback = orange!10
}{ex}

\makeatother
\newtcbtheorem[auto counter, number within = subsection]
{expln}{Expln}{%                                                        
  breakable,
  fonttitle = \bfseries,
  colframe = red!75!black,
  colback = red!10
}{exp}

\makeatother
\newtcbtheorem[auto counter, number within = subsection]
{ovr}{Overview}{%                                                        
  breakable,
  fonttitle = \bfseries,
  colframe = blue!75!black,
  colback = blue!10!white!50
}{ovr}


\makeatother
\newtcbtheorem[auto counter, number within = subsection]
{refer}{Reference}{%                                                        
  breakable,
  fonttitle = \bfseries,
  colframe = red!75!black,
  colback = red!10
}{refer}
\begin{document}

\makeatother
\newtcbtheorem[auto counter, number within = subsection]
{thm}{Thm}{%                                                        
  breakable,
  fonttitle = \bfseries,
  colframe = orange!75!black,
  colback = orange!10
}{thm}

% \setcounter{section}{1}
\maketitle
\tableofcontents
\break
% \section*{Lecture 1}

\section{Lecture - Feb 1}
lec06ProjG2D.pdf
\subsection{Homogenous Coordinates}
Homog rep of 2D points and lines:
\[
    ax+by+c = 0 \Rightarrow (a,b,c)^\top (x,y,1) = 0
\]
\noindent
point $\mathbf x$ lies on line $\mathbf l$ iff
\[
    \mathbf l^\top \mathbf x = \mathbf x^\top \mathbf l = 0
\]
Scale not inportant for incidence relation; equivalence class of vectors (any vector representative)
\[
    (a,b,c)^\top \sim k(a,b,c)^\top , \forall k \ne 0 \ \ \ \ \
    (x,y,1)^\top \sim k(x,y,1)^\top, \forall k \ne 0
\]
Set of all equivalence classes in $\mathbb R^3$ is $(0,0,0)^\top$, forms $\mathbb P^2$
\\
2 homogenous points are equivalent if they are collinear.\\
Homogenous coords - $(x_1, x_2, x_3)^\top$ (2DOF)\\
Inhomogenous coords - $(x,y)^\top$ \\[5pt]
$
\begin{bmatrix}
    x \\ y\\ 0
\end{bmatrix}
$ - point at infinity (ideal/vanishing point)
\\[10pt]
Intersection of two lines, $\mathbf l$ and $\mathbf l'$:
\[
    \mathbf x = \mathbf l \times \mathbf l'
\]
Line connecting two points, $\mathbf x$ and $\mathbf x'$:
\[
    \mathbf l = \mathbf x \times \mathbf x'
\]
\begin{expln}
{Cross Product}{}
\[
    \mathbf x \times \mathbf x' = {[\mathbf x]}_\times \mathbf x'
\]
Easier to represent cross product with:
\[
    \mathbf{[x]}_\times =\begin{bmatrix}
        0 & z & -y \\ 
        -z & 0 & x \\
        y & -x & 0
    \end{bmatrix}
\]
\end{expln}
\noindent
Intersection of Parallel Lines, $\mathbf l = (a,b,c)^\top$ and $\mathbf l' = (a,b,c')^\top$
\\
(parallel lines meet at ideal points in projective geometry; don't meet in Euclidean)
\[
    \mathbf l \times \mathbf l' = (b, -a, 0)^\top
\]
$(b,-a)$ - tangent vector; $(a,b)$ - normal direction
\\[5pt]
Ideal Point: \[
    \mathbf x_\infty = \begin{bmatrix}
        x_1 & x_2 & 0
    \end{bmatrix}^\top
    \]
Line at Infinity:
\[
    \mathbf l_\infty = \begin{bmatrix}
        0 & 0 & 1
    \end{bmatrix}^\top
\]
% \\
$\mathbb P^2 = \mathbb R^2 \cup \mathbf l_\infty$ 
($\mathbb P^2$ is $\mathbb R^2$ plus a "line at infinity")

\subsection{2D Projective Plane}
Perspective imaging models 2d projective space. Each 3D ray is a point in $\mathbb P^2$ (homogenous coords)
Homog coords:
\[
    \begin{bmatrix}
        x \\ y \\ z
    \end{bmatrix} = s \begin{bmatrix}
        x \\ y \\ z
    \end{bmatrix} \ \ \ \forall s \ne 0
\]
Inhomog Coords:
\[
    \begin{bmatrix}
        x' \\ y' 
    \end{bmatrix} = \frac{1}{z}
    \begin{bmatrix}
        x \\ y
    \end{bmatrix}
\]
principal axis - center ray 

\subsection{Lines}
Plane through origin projects to mathetmatical line when it is intersected w/ img plane
\\
Projective Line - a plane through the origin
\[
    \mathbf l ^\top \mathbf x = \mathbf x ^\top \mathbf l = 0
\]
Ideal Line - plane parallel to img
\[
    \mathbf l_\infty = \begin{bmatrix}
        0 \\ 0\\ 1
    \end{bmatrix}
\]
\begin{expln}
    {Duality}{}
    For any 2D projective property, dual property holds when the role of points 
    and lines are interchanged \\
    (symmetrical)
\end{expln}
\pagebreak
\subsection{Conics}
Curve described by 2nd degree equation in plane:
\[
    ax^2 +bxy + cy^2 + dx +ey +f = 0
\]
or homogenized: $x\mapsto \frac{x_1}{x_3}, y\mapsto \frac{x_2}{x_3}$
\[
    ax_1^2 + bx_1x_2+cx_2^2+dx_1x_3+ex_2x_3 + fx_3^2 = 0
\] or in matrix form:
\[
    \mathbf x^\top \mathbf C \mathbf x = 0 \text{ with } \mathbf C =\begin{bmatrix}
        a & b/2 & d/2 \\
        b/2 & c & e/2 \\
        d/2 & e/2 & f
    \end{bmatrix} 
\]
5DOF (ratios b/w distinct elements)
\subsubsection{Find Conics}
5 points define a conic;
\[
    [x_i^2, x_i, y_i, y_i^2, x_i, y_i, 1] \cdot \mathbf c \ \ \ \ \ \mathbf c = [a,b,c,d,e,f]^\top
\]
Stacking contraints yields:
\[
    \begin{bmatrix}
        x_1^2 & x_1y_1 & y_1^2 & x_1 & y_1 & 1 \\[5pt]
        x_2^2 & x_2y_2 & y_2^2 & x_2 & y_2 & 1 \\[5pt]
        x_3^2 & x_3y_3 & y_3^2 & x_3 & y_3 & 1 \\[5pt]
        x_4^2 & x_4y_4 & y_4^2 & x_4 & y_4 & 1 \\[5pt]
        x_5^2 & x_5y_5 & y_5^2 & x_5 & y_5 & 1
    \end{bmatrix}\mathbf c = 0
\]

\subsubsection{Tangent Lines to Conics}
The line $\mathbf l$ tangent to $\mathbf C$ at point $\mathbf x$ on $\mathbf C$ is given
by $\mathbf l = \mathbf C \mathbf x$ 
\[
    \mathbf x_0 ^\top \mathbf C \mathbf x = 0
\]

\subsubsection{Dual Conics}
Line tangent to conic $\mathbf C$ statisfies $\mathbf l^\top \mathbf C^* \mathbf l =0$
\\
In general ($\mathbf C$ full rank) - $\mathbf C^* = C^{-1}$
\\
Dual conics = line conics = convic envelope
\subsubsection{Degneerate Conics}
Conic \textbf{degenerate} iff matrix $\mathbf C$ is not full rank
\\
Degenerate line conics: 2 points (rank 2), double point (rank 1)\\
For degen conics: $(\mathbf C^*)^* \ne \mathbf C$ 


\section{Lecture - Feb 6}
\subsection{Lines}
Line b/w 2 points is cross product of the points; point intersecting 2 lines 
is cross product of lines.
\\
Parallel lines (distinct at small scale) become indistinguishable at infinity.


\subsection{Projective Transformation}
\begin{definition}
    {Projectivity}{}
    Projectivity is an invertible mapping $h$ from $\mathbb P^2$ to itself s.t. 
    three points $\mathbf x_1,\mathbf x_2,\mathbf x_3$ lie on the same line iff $h(x_1), h(x_2), h(x_3)$
    do
\end{definition}

\begin{thm}
    {}{}
    Mapping $h: P^2 \to P^2$ is a projectivity iff there exists a 
    nonsingular $3\times 3$ matrix $H$ s.t. for any point in $P^2$ represented by a 
    vector $\mathbf x$, it is true that $h(\mathbf x) = H\mathbf x$
\end{thm}

% \begin{example}
%     {State Space Model}{}
%     Track corners, homography from corner
% \end{example}


% \subsection{Planar texture Variability}
% Proj. Var - computational issues when solving for all $\Delta$ params. (unstable)


% \subsection{Mapping b/w Planes}
% diff plane $\to$ diff numerical coord, but homography still exists 

% \subsection
\noindent
Transforming Lines:
\[
    \mathbf l ' = H^{-\top} \mathbf l
\]
% cond number is how collinear stuff is
\noindent
affine - parallelism\\
metric - orthogonalism\\
euclidian - absolute scale

% \subsubsection{Planar Projective Warping}


\subsubsection{Affinity and Projectivity on line at inf}
Affine preserves $0$ (stays at infinity)\\
Projectivity changes $0$, line at infinity potentionally becomes finite
\begin{thm}
    {}{}
    Line at infinity is a fixed line under a projective transformation $H$ iff
    $H$ is affinity.\\
    Affine transformation is one that preserves line at infinity
\end{thm}
\noindent
Rectification/post-processing to get line at infinity back (fix parallel lines) after
projective. 

\subsection{Affine Rectification}
Take img, get parallel lines, construct $v_1,v_2$ as intersections for 2 sets of 
parallel lines, intersescted points on $\mathbf l_\infty$
\[
    \mathbf l_\infty = [l_1, l_2, l_3]
\]
% \\
\noindent
Fix 2 DOF relating to projective geometry, undo projective part of transformation;
just get affine part 
\[
    H_{PA} = \begin{bmatrix}
        1 & 0 & 0 \\
        0 & 1 & 0 \\
        l_1 & l_2 & l_3
    \end{bmatrix} H_A
\]

\subsubsection{}
\[
    x_i' = Hx_i
\]\noindent
Want to show that the points are collinear
\[
   \lambda  
   \begin{bmatrix}
    x' \\ y' \\ w'
   \end{bmatrix} =
   \begin{bmatrix}
    h_{11} & h_{12} & h_{13} \\
    h_{21} & h_{22} & h_{23} \\
    h_{31} & h_{32} & h_{33}
   \end{bmatrix}
   \begin{bmatrix}
    x \\y \\ 1
   \end{bmatrix}
\] 
\noindent
2 independent eq / point; 8DOF for projective $\to$ 4 points needed


\subsection{DLT}
Estimate $H$; $Hx_i$ involves homogenous vectors
vectors, $H_ix$ and $x_i$ need only be in the same direction and not strictly equal
\[
    \begin{bmatrix}
        x \\ 1
    \end{bmatrix} \equiv \begin{bmatrix}
        \lambda x \\ \lambda
    \end{bmatrix}
\]
\noindent
Specify "same directionality" via cross product formulation
\[
    x_i' \times Hx_i = 0
\]
Conditioning of $A$ isn't good in general
\\
Minimizes
\[
    \| Ah \|
\]
$e = Ah$ (residual)

\subsubsection{Importance of Normalization}
Order of magnitudes different, close to singular (poorly conditioned).\\
Normalize so centroid is origin, rescale points to be w/in 1.
\\
Improve numerical problems. \\
Denormalize solution to get pixel coordinates.

\subsubsection{Degen. Config}
Avoid having more than 3 collinear points. \\
Typically want 4 points clustered


\subsection{Parameter Estimation}
trifocal tensor - relate point w/ 3 imgs (not covered in class)

\subsubsection{Solving LS}
if $m=n$ then
\[
    \mathbf x = \mathbf A^{-1} \mathbf b
\]
\noindent
if $m > n$, overconstrained, cant invert; use LSS (via QR or pseudoinverse)

\subsubsection{Homogenous Sys. of Eq}
Solve $A\mathbf x = 0$, trivial sol $\mathbf x=\vec 0$ but don't want to use this. Find other vals.
\\
Solve via SVD ($A = UDV^\top$), with $\mathbf x$ as the \textbf{last column} of $V$



\subsubsection{Inhomogenous sol}
assumes $h_9 = 1$, wont work if we want to have horizon in center 









\end{document}