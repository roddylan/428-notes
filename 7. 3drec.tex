\documentclass{article}
\usepackage{textcomp, gensymb}
\usepackage{utf8add}
\usepackage[most]{tcolorbox}
\usepackage{hyperref}
\usepackage{cleveref}
\usepackage{amsmath}
\usepackage{amssymb}
\usepackage{tcolorbox}
\usepackage{outlines}
\usepackage{enumitem}
\usepackage{graphics}
\usepackage{bbm}
\hypersetup{
    colorlinks,
    citecolor=black,
    filecolor=black,
    linkcolor=black,
    urlcolor=black
}
\DeclareMathOperator*{\argmax}{arg\,max}
\DeclareMathOperator*{\argmin}{arg\,min}

% \setenumerate[1]{label=\Roman*.}
% \setenumerate[2]{label=\Alph*.}
% \setenumerate[3]{label=\roman*.}
% \setenumerate[4]{label=\alph*.}

\title{CMPUT 428: 3D Modelling}
\author{Roderick Lan}
% \date{January 11, 2024}
\date{}

\usepackage{natbib}
\makeatletter
% \crefformat{tcb@cnt@Example}{example~#2#1#3}
% \Crefformat{tcb@cnt@Example}{Example~#2#1#3}
\makeatother
\newtcbtheorem[auto counter, number within = subsection]
{definition}{Definition}{%                                                        
  breakable,
  fonttitle = \bfseries,
  colframe = blue!75!black,
  colback = blue!10
}{def}

\makeatother
\newtcbtheorem[auto counter, number within = subsection]
{example}{Example}{%                                                        
  breakable,
  fonttitle = \bfseries,
  colframe = orange!75!black,
  colback = orange!10
}{ex}

\makeatother
\newtcbtheorem[auto counter, number within = subsection]
{expln}{Expln}{%                                                        
  breakable,
  fonttitle = \bfseries,
  colframe = red!75!black,
  colback = red!10
}{exp}

\makeatother
\newtcbtheorem[auto counter, number within = subsection]
{ovr}{Overview}{%                                                        
  breakable,
  fonttitle = \bfseries,
  colframe = blue!75!black,
  colback = blue!10!white!50
}{ovr}


\makeatother
\newtcbtheorem[auto counter, number within = subsection]
{refer}{Reference}{%                                                        
  breakable,
  fonttitle = \bfseries,
  colframe = red!75!black,
  colback = red!10
}{refer}
\begin{document}

\makeatother
\newtcbtheorem[auto counter, number within = subsection]
{thm}{Thm}{%                                                        
  breakable,
  fonttitle = \bfseries,
  colframe = orange!75!black,
  colback = orange!10
}{thm}

% \setcounter{section}{1}
\maketitle
\tableofcontents
\break
% \section*{Lecture 1}

\section{Lecture - Feb 29}
lec07modeling18EmbMedia.pdf\\[5pt]
3D from video is:
\begin{outline}
    \1 inexpensive
    \1 quick and convenient for the user
    \1 integrates with existing software (ie. maya, blender, c4d, ue4, etc.)
\end{outline}
\subsection{Prelims; Macro Geometry}
Shape from silhouettes:
\begin{list}{}{}
    \item more = better
    \item provides good starting approx. for object
\end{list}
Structure from motion: % photogrammetry
\begin{list}{}{}
    \item uses triangulations to get 3d points
    \item can be done with 'unstructured' imgs/scene (ie. from phone camera)
    % \item 
\end{list}


\subsection{Fundamental Problems}
\begin{enumerate}
    \item \textbf{Resection} of 3D camera matrices
    \item \textbf{Intersection}
    \item \textbf{Structure From Motion (SFM)}
\end{enumerate}

\subsubsection{Resection}
Projection Equation:
\[
    x_i = P_i X
\]
Resection:
\[
    x_i, X\to P
\]
Imagine you have $x$ (ie. 3d model) and image $x_i$ (projection of points on img). 
Finding camera = finding projection of rays that map 3d point to image point.
\\
Can have many potential camera planes that all map (?)


\subsubsection{Intersection}
Projection Equation:
\[
    x_i = P_i X
\]
Intersection:
\[
    x_i, P_i \to X
\]
Have $\ge 2$ cameras; projections in $\ge 2$ images. Intersect rays to get 3d points. 
\\
Need to 'know' the camera properties. 

\subsubsection{SFM}
Projection Equation:
\[
    x_i = P_i X
\]
Structure From Motion (SFM):
\[
    x_i \to P_i, X
\]
Get image predictions ($x_i$) from point tracking.
\\
Get image points in $\ge 2$ views, calculate 3d points (structure) and 
camera projection matrices (motion)
\\
Estimate projective structure\\
Rectify the reconstruction to metric (autocalibration)
\\
(dont know the cameras; figure out constraints)

\subsection{Seq. 3D Structure From Motion}
Use 2 and 3 view geometry
\begin{outline}[enumerate]
    \1 Initialize structure and motion from 2 views
    \1 for each additional view:
        \2 determine pose
        \2 refine and extend structure
    \1 determine correspondences robustly by jointly estimating matchies and epipolar constraints
\end{outline}
\subsection{2 View Geometry}
2 points in views $x_i$, comes from 3d point $X$. If you have camera center, know that ray 
from center to point extends to infinity; $X$ lies somewhere on ray. 
\\
line connecting camera centers
\\
epipole - intersecting point b/w line and image plane
\\
epipolar line - line b/w imaged $x_i$ and epipole
\subsubsection{Epipolar Geometry}
C, C' x, x' and X are coplanar
\\
Epipolar geometry exists independent of 3d point, only related to camera and 2d point
projections.
\\
% Dont know $X$, argue that 
There is a family of epipolar planes \\
Epipolar lines go towards parallel lines
\\
Family of planes and lines intersect infinitely far apart.
\\
Epipoles ($e, e'$) can be finite or infinite
\begin{list}{}{}
    \item intersection of baseline with img plane
    \item projection of projection center in other img
    \item vanishing point of camera motion direction
\end{list}
Epipolar plane - plane containing baseline (1D family)
Epipolar line - intersection of epipolar plane with image 

\begin{example}
    {Parallel Motion with img plane}{}
    Epipoles at infinity
\end{example}
% optic flow must lie on epipolar lines
% \begin{example}
%     {Forward motion}{}

% \end{example}

\subsection{Fundamental Matrix}
Algebraic representation of epipolar geometry.
\[
    \mathbf x \mapsto \mathbf l'
\]
This mapping is a singular correlation; represented by fundamental matrix $F$
\\
$3 \times 3$ matrix with 7 dof (8 independent ratios)

\subsubsection{Algebraic Derivation}
Let $x$ be $PX$
\begin{flalign*}
    P^{+}x &= P^{+}P X
    \intertext{Say that pseudoinverse represents line from center to point}
    \mathbf X(\lambda) &= \mathbf P^+ \mathbf x + \lambda \mathbf C \\
    \mathbf l' &= \mathbf{P'C} \times \mathbf{P'P^+x} \\
    F &= [e']_\times P'P^+
\end{flalign*}
$\lambda$ is a scalar
\\
$[e']_\times$ skew symmetric cross product matrix (rank 2)
\\
doesnt work for $C=C' \Rightarrow F=0$ (must have distinct camera centers)
\\[5pt]
Alternatively can write:
\[
    F = [e']_\times H_\infty
\]
\[
    H_\infty = K^{-1}RK
\]
\subsubsection{Geometric Derivation}
Step 1: $X$ on a plane $\pi$
\[
    x' = H_\pi x
\]
Step 2: Epipolar line $l'$ 
\[
    l' = e' \times x' = [e']_\times H_\pi x = Fx
\]
Map from 2D to 1D family (rank 2)
\\
(plane $\pi$ can be different for each point, \textbf{unrelated} to epipolar plane)

\subsubsection{Correspondence Condition}
Exists 2D matrix that ouputs epipolar line.
\\
If $\mathbf x'$ lies on $\mathbf l'$, $\mathbf{x' l'} = 0$ 
\[
    \mathbf{x' \cdot Fx} = 0
\]

\noindent\rule{\textwidth}{.5pt}
\\[15pt]
$\mathbf F$ is the unique $3 \times 3$ rank 2 matrix that satisfies 
\[
    \mathbf{x'^\top Fx} = 0 \ \ \ \ \forall \mathbf x\leftrightarrow \mathbf x'
\]
\begin{enumerate}
    \item \textbf{Transpose}: if $\mathbf F$ is a fundamental matrix for (P, P'), $\mathbf F^\top$
        is fundamental matrix for (P', P)
    \item Epipolar Lines: $\mathbf{l' = Fx}$ and $\mathbf{l=F^\top x'}$
    \item Epipoles: on all epipolar lines, $\mathbf{e'^\top Fx = 0}$ $\forall \mathbf x, 
    \Rightarrow e'^\top F=0$, similarly $\mathbf{Fe}= 0$
    \item $\mathbf F$ has 7dof ($3\times 3$ - 1 (homog), - 1 (rank 2))
    \item $\mathbf F$ is a correlation, projective mapping from a point $\mathbf x$ 
        to a line $\mathbf{l' = Fx}$ (not a proper corr., ie. not invertible)
\end{enumerate}

\subsubsection{Summary}
Algebraic rep. of epipolar geom.
\\
Step 1: $\mathbf X$ on plane $\pi$
\[
    \mathbf{x' = Hx}
\]
Step 2: epipolar line $\mathbf l'$
\begin{flalign*}
    \mathbf{l'} &= \mathbf{e' \times x' = [e']_\times x'} \\ &= \mathbf{[e']_\times Hx = Fx}
\end{flalign*}
\[
    \mathbf{x'^\top Fx =0 }
\]

\subsubsection{Computing F}
Use 8 points
\[
    \mathbf{x'^\top Fx} =0 
\]
(SLIDE 43-44)

\end{document}

