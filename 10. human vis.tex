\documentclass{article}
\usepackage{textcomp, gensymb}
\usepackage{utf8add}
\usepackage[most]{tcolorbox}
\usepackage{hyperref}
\usepackage{cleveref}
\usepackage{amsmath}
\usepackage{amssymb}
\usepackage{tcolorbox}
\usepackage{outlines}
\usepackage{enumitem}
\usepackage{graphics}
\usepackage{bbm}
% \usepackage{hhline}

\hypersetup{
    colorlinks,
    citecolor=black,
    filecolor=black,
    linkcolor=black,
    urlcolor=black
}
\DeclareMathOperator*{\argmax}{arg\,max}
\DeclareMathOperator*{\argmin}{arg\,min}

% \setenumerate[1]{label=\Roman*.}
% \setenumerate[2]{label=\Alph*.}
% \setenumerate[3]{label=\roman*.}
% \setenumerate[4]{label=\alph*.}

\title{CMPUT 428: 3D Modeling}
\author{Roderick Lan}
% \date{January 11, 2024}
\date{}

\usepackage{natbib}
\makeatletter
% \crefformat{tcb@cnt@Example}{example~#2#1#3}
% \Crefformat{tcb@cnt@Example}{Example~#2#1#3}
\makeatother
\newtcbtheorem[auto counter, number within = subsection]
{definition}{Definition}{%                                                        
  breakable,
  fonttitle = \bfseries,
  colframe = blue!75!black,
  colback = blue!10
}{def}

\makeatother
\newtcbtheorem[auto counter, number within = subsection]
{example}{Example}{%                                                        
  breakable,
  fonttitle = \bfseries,
  colframe = orange!75!black,
  colback = orange!10
}{ex}

\makeatother
\newtcbtheorem[auto counter, number within = subsection]
{expln}{Expln}{%                                                        
  breakable,
  fonttitle = \bfseries,
  colframe = red!75!black,
  colback = red!10
}{exp}

\makeatother
\newtcbtheorem[auto counter, number within = subsection]
{ovr}{Overview}{%                                                        
  breakable,
  fonttitle = \bfseries,
  colframe = blue!75!black,
  colback = blue!10!white!50
}{ovr}


\makeatother
\newtcbtheorem[auto counter, number within = subsection]
{refer}{Reference}{%                                                        
  breakable,
  fonttitle = \bfseries,
  colframe = red!75!black,
  colback = red!10
}{refer}
\begin{document}

\makeatother
\newtcbtheorem[auto counter, number within = subsection]
{thm}{Thm}{%                                                        
  breakable,
  fonttitle = \bfseries,
  colframe = orange!75!black,
  colback = orange!10
}{thm}

% \setcounter{section}{1}
\maketitle
\tableofcontents
\break
% \section{}
\section*{Final Lecture}
Goal of computer vision
\section{Eye}
6 muscles that adjust

\section{Retina}
Light only detected at back of retina
\\
Photons refract before it reaches back of retina, eyes need to recalibrate
how it processes signals

\section{Photoreceptors}
Rods - night vision; no color
\begin{list}{}{}
  \item 125 mill, none in fovea
  \item 20 rods : 1 cone
\end{list}
Cones - color sensitive; poor light sens
\begin{list}{}{}
  \item 6.4 mill, peak density in fovea
\end{list}

\section{How the eye works}
large molecules, need large geometric changes to activate
\\
cis retinal - low energy
\\
trans retinal - slightly higher energy
\[
  \text{cis retinal} \xrightarrow{\text{energy from incoming light photon}}
  \text{trans retinal}
\]


\section{Interneurons and Ganglion cells}


\section{Visual system adjusts itself}
Adjusts both in time and space.
\\
Adjusts for:
\begin{list}{}{}
  \item light sensitivity / gain
  \item neural fatigue
  \item adjustient of priors (opposite dir.)
  \item error correction
\end{list}

\section{Color opponent ganglion cells}

\section{LGN}
high motion and fast things \\
switchboard b/w retina and visual cortex

\section{Simple Cells in V1}
Direction sensitive "line finders"


\section{Receptive field}

\section{Dorsal + Ventral}
Dorsal - spatial vision
\\
Ventral - object recognition


\section{Boundary effect}

\section{Attention}


\end{document}
