\documentclass{article}
\usepackage{textcomp, gensymb}
\usepackage{utf8add}
\usepackage[most]{tcolorbox}
\usepackage{hyperref}
\usepackage{cleveref}
\usepackage{amsmath}
\usepackage{amssymb}
\usepackage{tcolorbox}
\usepackage{outlines}
\usepackage{enumitem}
\usepackage{graphics}
\usepackage{bbm}
\hypersetup{
    colorlinks,
    citecolor=black,
    filecolor=black,
    linkcolor=black,
    urlcolor=black
}
\DeclareMathOperator*{\argmax}{arg\,max}
\DeclareMathOperator*{\argmin}{arg\,min}

\setenumerate[1]{label=\Roman*.}
\setenumerate[2]{label=\Alph*.}
\setenumerate[3]{label=\roman*.}
\setenumerate[4]{label=\alph*.}

\title{CMPUT 428: Visual Servoing}
\author{Roderick Lan}
% \date{January 11, 2024}
\date{}

\usepackage{natbib}
\makeatletter
% \crefformat{tcb@cnt@Example}{example~#2#1#3}
% \Crefformat{tcb@cnt@Example}{Example~#2#1#3}
\makeatother
\newtcbtheorem[auto counter, number within = subsection]
{definition}{Definition}{%                                                        
  breakable,
  fonttitle = \bfseries,
  colframe = blue!75!black,
  colback = blue!10
}{def}

\makeatother
\newtcbtheorem[auto counter, number within = subsection]
{example}{Example}{%                                                        
  breakable,
  fonttitle = \bfseries,
  colframe = orange!75!black,
  colback = orange!10
}{ex}

\makeatother
\newtcbtheorem[auto counter, number within = subsection]
{expln}{Expln}{%                                                        
  breakable,
  fonttitle = \bfseries,
  colframe = red!75!black,
  colback = red!10
}{exp}

\makeatother
\newtcbtheorem[auto counter, number within = subsection]
{ovr}{Overview}{%                                                        
  breakable,
  fonttitle = \bfseries,
  colframe = blue!75!black,
  colback = blue!10!white!50
}{ovr}


\makeatother
\newtcbtheorem[auto counter, number within = subsection]
{refer}{Reference}{%                                                        
  breakable,
  fonttitle = \bfseries,
  colframe = red!75!black,
  colback = red!10
}{refer}
\begin{document}

\makeatother
\newtcbtheorem[auto counter, number within = subsection]
{thm}{Thm}{%                                                        
  breakable,
  fonttitle = \bfseries,
  colframe = orange!75!black,
  colback = orange!10
}{thm}

% \setcounter{section}{1}
\maketitle
\tableofcontents
\break
% \section*{Lecture 1}

\section{Lecture - Feb 8}
lec10VisServEmbVid.pdf
\subsection{Vision Based Control (Visual Servoing)}
Robot acting in Euclidean space, gonna be in some manifold of projective space.
\\
4 points give homogrpahy, homography not necessarily euclidean, leverage the fact that the 
top is in euclidean space, \dots
\\
If full motion isnt convex, there exists some subdivision of motion that is convex $\to$ 
intermediate goal points w/ convex subsections of motion.

\subsection{Problems}
Chaining transforms $\to$ accumulate errors
% \\
% Can formulate ()
% \\
\subsection{}
Use broyden's method for optimization (dont need to calc. deriv.)
\\
error $y=f(x)$ is visual error, assume it is smooth convex func. \\
Getting Jacobian: can always get discrete deriv 
(ie. deriv b/w frames in optical flow)
\begin{list}{}{}
    \item move joint 1 (up to $10\degree$), get all partial derivs for joint
    \item do same for other joints
    \item fill in jacobian 
\end{list}
\subsection{Find J Method 2}
For every motion, jacobian should obey "secant constraint"
\begin{list}{}{}
    \item get 'constraints' from joint motion
    \item stack into matrix, fit $J$
\end{list}

\subsection{Find J Method 3}
Recursive Secant constraints\\
iterative secant update for jacobian (Broyden)


\subsection{Specifications}
image encoding $E(y) =0$\\
Guarantee that you're actually solving the problem (ie. objects are actually touching
and don't just look like they are due to perspective)
\\
(task ambiguity)






\end{document}